\section*{Exercises \#1}
\begin{enumerate}[\bfseries 1.]
	\item (Rotation Matrix in the Complex Plane). 
	Let \(\phi\in\mathbb{R}\) and consider the \(2\times2\) matrix  
	\[
	A \;=\;
	\begin{pmatrix}
		\cos\phi & -\sin\phi\\[6pt]
		\sin\phi & \cos\phi
	\end{pmatrix}\!\!.
	\]  
	\begin{enumerate}[(a)]
		\item Prove that \(A\) is a unitary operator on \(\mathbb{C}^2\); that is, show $
		A^\dagger A \;=\; I_2$, where \(A^\dagger\) is the conjugate-transpose of \(A\).
		\item Determine the full spectrum of \(A\) and exhibit for each eigenvalue a corresponding (nonzero) eigenvector.
	\end{enumerate}
\begin{proof}[\normalfont\bfseries\textcolor{magenta}{Sol}] 
\begin{enumerate}[(a)]
	\item Since \(A\) has only real entries, \(A^\dagger = A^T\). Then \begin{align*}
		A^T A
		&=\begin{pmatrix}
			\cos\phi & \sin\phi\\
			-\sin\phi & \cos\phi
		\end{pmatrix}
		\begin{pmatrix}
			\cos\phi & -\sin\phi\\
			\sin\phi & \cos\phi
		\end{pmatrix}\\
		&=\begin{pmatrix}
			\cos^2\phi + \sin^2\phi & -\cos\phi\,\sin\phi + \sin\phi\,\cos\phi\\[6pt]
			-\sin\phi\,\cos\phi + \cos\phi\,\sin\phi & \sin^2\phi + \cos^2\phi
		\end{pmatrix} \\
		&=\begin{pmatrix}
			1 & 0\\
			0 & 1
		\end{pmatrix}.
	\end{align*}
	\item We need to find \(\lambda\in\mathbb C\) and nonzero \(\vec{v}=(v_1,v_2)\) such that $
	A\,\vec{v}=\lambda\,\vec{v}$. Equivalently, \[
	(A-\lambda I_2)\vec{v}=\begin{pmatrix}
		\cos\phi -\lambda & -\sin\phi\\
		\sin\phi & \cos\phi -\lambda
	\end{pmatrix}
	\begin{pmatrix}v_1\\v_2\end{pmatrix}
	=\begin{pmatrix}0\\0\end{pmatrix}.
	\] Nontrivial solutions exist exactly when \[
	\det\bigl(A-\lambda I\bigr)
	=(\cos\phi-\lambda)^2 + \sin^2\phi =0.
	\] Then
	\[
	(\cos\phi-\lambda)^2 + \sin^2\phi
	=\lambda^2 -2\lambda\cos\phi + (\cos^2\phi+\sin^2\phi)
	=\lambda^2 -2\lambda\cos\phi +1,
	\] and so \begin{align*}
	\lambda^2 -2(\cos\phi)\,\lambda +1 =0 &\implies
	\lambda \;=\;\frac{2\cos\phi\pm\sqrt{(2\cos\phi)^2-4\cdot1\cdot1}}{2} \\
	&\implies\lambda=\cos\phi\;\pm\;\sqrt{\cos^2\phi-1}\\
	&\implies\lambda=\cos\phi\;\pm\;\sqrt{-\sin^2\phi}\\
	&\implies\lambda = \cos\phi \pm i\sin\phi
	= e^{\pm i\phi}.
	\end{align*} 
	Thus the two eigenvalues are
	\[
	\lambda_1 = e^{i\phi},\qquad \lambda_2 = e^{-i\phi}.
	\]Since \begin{align*}
		(A-\lambda_1 I)\vec{v}=0
		\implies&\; \begin{pmatrix}
			\cos\phi -(\cos\phi+i\sin\phi) & -\sin\phi\\[6pt]
			\sin\phi & \cos\phi -(\cos\phi+i\sin\phi)
		\end{pmatrix}
		\begin{pmatrix}v_1\\v_2\end{pmatrix}
		=0\\
		\implies&\; \begin{cases}
			(-\,i\sin\phi)\,v_1 \;-\;\sin\phi\,v_2 =0,\\
			\sin\phi\,v_1 \;-\;(i\sin\phi)\,v_2 =0.
		\end{cases}
		\implies\; \begin{cases}
		-iv_1 \;-v_2 =0,\\
	v_1 \;-iv_2 =0.
	\end{cases}\quad\text{if $\sin\phi\neq0$}\\
	\implies&\vec{v}=t\begin{pmatrix}
		1 \\ -i
	\end{pmatrix}\quad\text{with}\; t\neq 0,
	\end{align*}
	a normalized eigenvector is $\displaystyle
	\ket{v_1} = \frac1{\sqrt{2}}\begin{pmatrix}1\\[-3pt]-\,i\end{pmatrix}.$ Similarly, we have $\displaystyle
	\ket{v_2} = \frac1{\sqrt{2}}\begin{pmatrix}1\\[-3pt]i\end{pmatrix}.$
\end{enumerate}
\end{proof}	
	\item (Action of the Kronecker Product on Tensor‐Product Vectors).
	Let  
	\[
	A\in\mathbb{C}^{n\times n},\quad B\in\mathbb{C}^{m\times m},
	\qquad
	\alpha\in\mathbb{C}^n,\quad \beta\in\mathbb{C}^m,
	\]  
	and form their Kronecker products \(A\otimes B\in\mathbb{C}^{(nm)\times(nm)}\) and \(\alpha\otimes\beta\in\mathbb{C}^{nm}\).  Show that  
	\[
	(A\otimes B)\,(\alpha\otimes\beta)
	\;=\;
	(A\,\alpha)\;\otimes\;(B\,\beta).
	\]
\begin{proof}[\normalfont\bfseries\textcolor{magenta}{Sol}]
1. Let \(\{e_i\}_{i=1}^n\) be the standard basis of \(\mathbb{C}^n\) and \(\{f_j\}_{j=1}^m\) the standard basis of \(\mathbb{C}^m\).  By definition of the tensor (Kronecker) product we have the basis  
\[
\{\,e_i\otimes f_j : 1\le i\le n,\;1\le j\le m\}\quad\text{for $\mathbb{C}^n\otimes\mathbb{C}^m\cong\mathbb{C}^{nm}$.}
\] Write
\[
\alpha = \sum_{i=1}^n \alpha_i\,e_i,
\qquad
\beta  = \sum_{j=1}^m \beta_j\,f_j.
\]
Then by bilinearity of the tensor product,
\[
\alpha\otimes\beta
= \sum_{i=1}^n\sum_{j=1}^m 
\bigl(\alpha_i\,\beta_j\bigr)\,(e_i\otimes f_j).
\] By the definition of the Kronecker‐product operator, \[
(A\otimes B)\bigl(e_i\otimes f_j\bigr)
= \bigl(A\,e_i\bigr)\;\otimes\;\bigl(B\,f_j\bigr),
\]
and linearity then gives \[
(A\otimes B)\,(\alpha\otimes\beta)
= \sum_{i=1}^n\sum_{j=1}^m \alpha_i\,\beta_j\,
(A\otimes B)\bigl(e_i\otimes f_j\bigr)
= \sum_{i,j} \alpha_i\,\beta_j\,
\bigl(Ae_i\otimes Bf_j\bigr).
\]
Observe that \[
A\alpha
= A\bigl(\sum_i \alpha_i e_i\bigr)
= \sum_i \alpha_i \,(A\,e_i),
\quad
B\beta
= \sum_j \beta_j \,(B\,f_j).
\]
Hence \[
A\alpha\;\otimes\;B\beta
= \biggl(\sum_i\alpha_i\,Ae_i\biggr)
\otimes
\biggl(\sum_j\beta_j\,Bf_j\biggr)
= \sum_{i,j} \alpha_i\,\beta_j\,
\bigl(Ae_i\otimes Bf_j\bigr)=(A\otimes B)\,(\alpha\otimes\beta).
\]
\end{proof}
	\item % problem 3
	\item % problem 4
	\newpage
	\item (SWAP Gate via Three CNOTs)\;% problem 5
	Let \[
	\mathrm{SWAP}\colon \mathbb{C}^2\otimes\mathbb{C}^2 \to \mathbb{C}^2\otimes\mathbb{C}^2
	\]
	be the two‐qubit operator defined on the computational basis by
	\[
	\mathrm{SWAP}\,\ket{x,y} \;=\; \ket{y,x},
	\quad x,y\in\{0,1\}.
	\]
	\begin{enumerate}[(i)]
		\item Prove that
		\[
		\mathrm{SWAP}
		= \mathrm{CNOT}_{1\to2}\;\circ\;\mathrm{CNOT}_{2\to1}\;\circ\;\mathrm{CNOT}_{1\to2},
		\]
		where $\mathrm{CNOT}_{i\to j}$ denotes a CNOT gate with control qubit $i$ and target qubit $j$.
		\item Show that the following circuit indeed effects the swap of the two qubits:
		\begin{center}
		\begin{quantikz}[column sep=1cm]
			\lstick{$\ket{x}$} & \ctrl{1} & \targ{}    & \ctrl{1} & \qw \\
			\lstick{$\ket{y}$} & \targ{}    & \ctrl{-1} & \targ{}  & \qw
		\end{quantikz}
	\end{center}
	\end{enumerate}
\begin{proof}[\normalfont\bfseries\textcolor{magenta}{Sol}]
	Since both \(U\) and \(\mathrm{SWAP}\) are unitary operators on the two‐qubit Hilbert space, it suffices to check their action on the computational basis \(\{\ket{x,y}:x,y\in\{0,1\}\}\).  Write
	\[
	\mathrm{CNOT}_{1\to2}\,\ket{x,y}
	=\ket{x,\;x\oplus y}, 
	\quad
	\mathrm{CNOT}_{2\to1}\,\ket{a,b}
	=\ket{a\oplus b,\;b},
	\]
	where \(\oplus\) denotes addition modulo~2.	
	
	\medskip
	\noindent\textbf{Step 1:} 
	Apply the first gate:\; $\ket{x,y}
	\;\xmapsto{\;\mathrm{CNOT}_{1\to2}\;}
	\ket{x,\;x\oplus y}.$
	
	\noindent\textbf{Step 2:} Apply the second gate, \(\mathrm{CNOT}_{2\to1}\), to the intermediate state:
	\[
	\ket{x,\;x\oplus y}
	\;\xmapsto{\;\mathrm{CNOT}_{2\to1}\;}
	\ket{\;x\oplus(x\oplus y),\;x\oplus y}
	=\ket{y,\;x\oplus y}.
	\]
	
	\noindent\textbf{Step 3:} Finally apply \(\mathrm{CNOT}_{1\to2}\) again:
	\[
	\ket{y,\;x\oplus y}
	\;\xmapsto{\;\mathrm{CNOT}_{1\to2}\;}
	\ket{y,\;y\oplus(x\oplus y)}
	=\ket{y,\;x}.
	\]
	
	\smallskip
	
	Hence the composite action on an arbitrary basis vector is
	\[
	U\,\ket{x,y}
	=\ket{y,x},
	\]
	which by definition is exactly \(\mathrm{SWAP}\,\ket{x,y}\).
\end{proof}
	\newpage
	\item (Matrix Representation of the Toffoli (CCX) Gate)\;% problem 6
	\begin{center}
		\begin{quantikz}[column sep=1cm]
			\lstick{$q_0$} & \targ{} & \qw\rstick{$q_0\oplus q_1q_2$} \\
			\lstick{$q_1$} & \ctrl{-1} & \qw\rstick{$q_1$} \\
			\lstick{$q_2$} & \ctrl{-1}  & \qw\rstick{$q_2$}
		\end{quantikz}
	\end{center}
	The three‐qubit Toffoli gate (also called the Controlled-Controlled-NOT, or CCX, gate) with qubits \(q_0,q_1\) as controls and \(q_2\) as target acts on the computational basis by
	\[
	\ket{q_0q_1q_2}
	\;\xrightarrow{CCX_{210}}\;
	\ket{(q_0\oplus (q_1\land q_2))q_1q_2},
	\qquad q_0,q_1,q_2\in\{0,1\}.
	\]
	Using the lexicographic ordering
	\(\ket{000},\ket{001},\ket{010},\dots,\ket{111}\),
	represent \(\mathrm{CCX}_{210}\) as an \(8\times8\) unitary matrix.
	
	Let the three-qubit Toffoli gate (also called the Controlled-Controlled-NOT, or CCX, gate) act on the computational basis 
	\(\{\ket{q_0q_1q_2}:q_i\in\{0,1\}\}\) by flipping the target qubit \(q_2\) if and only if both control qubits \(q_0\) and \(q_1\) are in state \(\ket{1}\).  
	
	\begin{enumerate}[(i)]
		\item Write down the action of CCX on each basis vector:
		\[
		\mathrm{CCX}\,\ket{q_0q_1q_2}
		= 
		\ket{q_0\,q_1\,\bigl(q_2 \oplus (q_0\land q_1)\bigr)}.
		\]
		\item Using the standard ordered basis
		\(\ket{000},\ket{001},\ket{010},\ket{011},\ket{100},\ket{101},\ket{110},\ket{111}\),
		represent \(\mathrm{CCX}\) as an \(8\times8\) unitary matrix.
	\end{enumerate}
\begin{proof}[\normalfont\bfseries\textcolor{magenta}{Sol}]
	With the lexicographic ordering
	\(\ket{000},\ket{001},\ket{010},\ket{011},\ket{100},\ket{101},\ket{110},\ket{111}\),
	its matrix representation is the \(8\times8\) unitary
	\[
	\mathrm{CCX}
	=
	\begin{pmatrix}
		1&0&0&0&0&0&0&0\\
		0&1&0&0&0&0&0&0\\
		0&0&1&0&0&0&0&0\\
		0&0&0&1&0&0&0&0\\
		0&0&0&0&1&0&0&0\\
		0&0&0&0&0&1&0&0\\
		0&0&0&0&0&0&0&1\\
		0&0&0&0&0&0&1&0
	\end{pmatrix},
	\]
	i.e.\ the first six basis states are fixed and the last two are swapped.
\end{proof}
\end{enumerate}
