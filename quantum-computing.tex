%\documentclass[12pt]{article}
%
%% Load essential packages for advanced math and formatting
%\usepackage{amsmath, amssymb, amsfonts, amsthm}
%\usepackage{physics}      % Provides convenient bra-ket notation macros
%\usepackage{geometry}       % Adjust page margins
%\usepackage{hyperref}       % Hyperlinks in the PDF
%\usepackage{enumitem}       % Better list control
%\usepackage{bm}             % Bold math symbols
%
%\geometry{margin=1in}
%
%% Custom macros for Dirac notation and common symbols
%\renewcommand{\ket}[1]{\left| #1 \right\rangle}     % ket vector
%\renewcommand{\bra}[1]{\left\langle #1 \right|}       % bra vector
%\renewcommand{\braket}[2]{\left\langle #1 \middle| #2 \right\rangle}  % inner product
%\renewcommand{\norm}[1]{\left\lVert #1 \right\rVert}   % norm
%
%% Theorem-like environments (if needed)
%\newtheorem{definition}{Definition}[section]
%\newtheorem{theorem}{Theorem}[section]
%
%\begin{document}
%	
%	\title{\vspace{-2em}Lecture Notes on Quantum Mechanics and Linear Algebra\\[0.5em]}
%	\author{Author Name}
%	\date{\today}
%	\maketitle
%	
%	\tableofcontents
%	\newpage
%	
%	%%%%%%%%%%%%%%%%%%%%%%%%%%%%%%%%%%%%%%%%%%%%%%%%%%%%%%%%%%%%%%%%%%%%%%
%	\section{Introduction}
%	%%%%%%%%%%%%%%%%%%%%%%%%%%%%%%%%%%%%%%%%%%%%%%%%%%%%%%%%%%%%%%%%%%%%%%
%	
%	Quantum Mechanics and Linear Algebra lie at the core of modern physics and mathematics. In these notes, we develop a rigorous framework for quantum states, Hilbert spaces, and the linear operators acting on them. As is well known, a \emph{quantum state} is represented as a vector in a Hilbert space , and many physical phenomena, such as superposition, are naturally described within this vector–space setting. For instance, the oft-quoted Schrödinger's cat example illustrates that a state may be in a superposition of ``alive'' and ``dead''.
%	
%	%%%%%%%%%%%%%%%%%%%%%%%%%%%%%%%%%%%%%%%%%%%%%%%%%%%%%%%%%%%%%%%%%%%%%%
%	\section{Quantum States and Hilbert Spaces}
%	%%%%%%%%%%%%%%%%%%%%%%%%%%%%%%%%%%%%%%%%%%%%%%%%%%%%%%%%%%%%%%%%%%%%%%
%	
%	\begin{definition}[Hilbert Space]
%		A \textbf{Hilbert space} $\mathcal{H}$ is a complete vector space over the complex numbers $\mathbb{C}$ equipped with an inner product 
%		\[
%		\braket{\phi}{\psi} : \mathcal{H} \times \mathcal{H} \to \mathbb{C},
%		\]
%		which induces the norm
%		\[
%		\norm{\psi} = \sqrt{\braket{\psi}{\psi}}.
%		\]
%	\end{definition}
%	
%	A quantum state of a physical system is a vector $\ket{\psi} \in \mathcal{H}$. The principles of quantum mechanics are deeply intertwined with the structure of $\mathcal{H}$.
%	
%	%%%%%%%%%%%%%%%%%%%%%%%%%%%%%%%%%%%%%%%%%%%%%%%%%%%%%%%%%%%%%%%%%%%%%%
%	\section{Dirac's Bra-Ket Notation}
%	%%%%%%%%%%%%%%%%%%%%%%%%%%%%%%%%%%%%%%%%%%%%%%%%%%%%%%%%%%%%%%%%%%%%%%
%	
%	Dirac's notation offers an elegant way to denote vectors and linear functionals in a Hilbert space. For any state $\ket{\psi} \in \mathcal{H}$, its dual vector is written as $\bra{\psi}$, and the inner product is expressed as
%	\[
%	\braket{\phi}{\psi} = \bra{\phi}\ket{\psi}.
%	\]
%	Linearity is expressed by the relation:
%	\[
%	z_1\ket{v_1} + z_2\ket{v_2} \in \mathcal{H}, \quad \text{with } z_1,z_2 \in \mathbb{C}.
%	\]
%	This notation not only simplifies the representation of states but also the operations on them.
%	
%	%%%%%%%%%%%%%%%%%%%%%%%%%%%%%%%%%%%%%%%%%%%%%%%%%%%%%%%%%%%%%%%%%%%%%%
%	\section{Inner Products and Orthogonality}
%	%%%%%%%%%%%%%%%%%%%%%%%%%%%%%%%%%%%%%%%%%%%%%%%%%%%%%%%%%%%%%%%%%%%%%%
%	
%	The inner product $\braket{\phi}{\psi}$ has the following properties:
%	\begin{itemize}[leftmargin=2em]
%		\item \textbf{Conjugate Symmetry:} \(\braket{\phi}{\psi} = \braket{\psi}{\phi}^*\).
%		\item \textbf{Linearity in the Second Argument:} For any \(z \in \mathbb{C}\),
%		\[
%		\braket{\phi}{z\psi} = z\braket{\phi}{\psi}.
%		\]
%		\item \textbf{Positivity:} \(\braket{\psi}{\psi} \geq 0\), with equality if and only if \(\ket{\psi} = 0\).
%	\end{itemize}
%	
%	Two vectors \(\ket{\phi}\) and \(\ket{\psi}\) are said to be \emph{orthogonal} if 
%	\[
%	\braket{\phi}{\psi} = 0.
%	\]
%	
%	%%%%%%%%%%%%%%%%%%%%%%%%%%%%%%%%%%%%%%%%%%%%%%%%%%%%%%%%%%%%%%%%%%%%%%
%	\section{Linear Independence and Basis}
%	%%%%%%%%%%%%%%%%%%%%%%%%%%%%%%%%%%%%%%%%%%%%%%%%%%%%%%%%%%%%%%%%%%%%%%
%	
%	A set of vectors \(\{\ket{v_1}, \ket{v_2}, \dots, \ket{v_n}\}\) in \(\mathcal{H}\) is said to be \emph{linearly independent} if the equation
%	\[
%	z_1\ket{v_1} + z_2\ket{v_2} + \cdots + z_n\ket{v_n} = 0
%	\]
%	implies \(z_1 = z_2 = \cdots = z_n = 0\) (with \(z_i \in \mathbb{C}\)). When any vector in \(\mathcal{H}\) can be written as a linear combination of these vectors, the set forms a \emph{basis}. An \emph{orthonormal basis} additionally satisfies:
%	\[
%	\braket{v_i}{v_j} = \delta_{ij},
%	\]
%	where \(\delta_{ij}\) is the Kronecker delta.
%	
%	The completeness relation in this context is given by:
%	\[
%	\sum_i \ket{v_i}\bra{v_i} = I,
%	\]
%	where \(I\) is the identity operator.
%	
%	%%%%%%%%%%%%%%%%%%%%%%%%%%%%%%%%%%%%%%%%%%%%%%%%%%%%%%%%%%%%%%%%%%%%%%
%	\section{Matrix Representation of States and Operators}
%	%%%%%%%%%%%%%%%%%%%%%%%%%%%%%%%%%%%%%%%%%%%%%%%%%%%%%%%%%%%%%%%%%%%%%%
%	
%	Any vector \(\ket{\psi}\) in a finite-dimensional Hilbert space can be expanded as:
%	\[
%	\ket{\psi} = \sum_{i=1}^{n} \psi_i \ket{v_i}, \quad \psi_i = \braket{v_i}{\psi}.
%	\]
%	In this basis, the ket is represented by a column vector:
%	\[
%	\ket{\psi} \;\widehat{=}\; 
%	\begin{pmatrix}
%		\psi_1 \\ \psi_2 \\ \vdots \\ \psi_n
%	\end{pmatrix},
%	\]
%	and the corresponding bra is the conjugate transpose (row vector):
%	\[
%	\bra{\psi} \;\widehat{=}\; 
%	\begin{pmatrix}
%		\psi_1^* & \psi_2^* & \cdots & \psi_n^*
%	\end{pmatrix}.
%	\]
%	
%	A linear operator \(\mathcal{L}\) acting on \(\mathcal{H}\) has a matrix representation with elements defined by:
%	\[
%	L_{ij} = \bra{v_i}\mathcal{L}\ket{v_j}.
%	\]
%	Its action on a state is then given by standard matrix multiplication.
%	
%	%%%%%%%%%%%%%%%%%%%%%%%%%%%%%%%%%%%%%%%%%%%%%%%%%%%%%%%%%%%%%%%%%%%%%%
%	\section{Change of Basis}
%	%%%%%%%%%%%%%%%%%%%%%%%%%%%%%%%%%%%%%%%%%%%%%%%%%%%%%%%%%%%%%%%%%%%%%%
%	
%	Consider two orthonormal bases \(\{\ket{v_i}\}\) and \(\{\ket{w_i}\}\) of \(\mathcal{H}\). A state \(\ket{\psi}\) can be written in either basis:
%	\[
%	\ket{\psi} = \sum_{i} \psi_i \ket{v_i} = \sum_{i} \phi_i \ket{w_i}.
%	\]
%	The transformation between the coefficients is given by a unitary matrix \(U\) such that
%	\[
%	\phi_i = \sum_{j} U_{ij}\psi_j,
%	\]
%	with the property that
%	\[
%	U^\dagger U = UU^\dagger = I.
%	\]
%	
%	%%%%%%%%%%%%%%%%%%%%%%%%%%%%%%%%%%%%%%%%%%%%%%%%%%%%%%%%%%%%%%%%%%%%%%
%	\section{Linear Operators and Their Properties}
%	%%%%%%%%%%%%%%%%%%%%%%%%%%%%%%%%%%%%%%%%%%%%%%%%%%%%%%%%%%%%%%%%%%%%%%
%	
%	A \emph{linear operator} \(\mathcal{L}:\mathcal{H}\to\mathcal{H}\) satisfies
%	\[
%	\mathcal{L}(a\ket{\psi}+b\ket{\phi}) = a\mathcal{L}\ket{\psi} + b\mathcal{L}\ket{\phi}, \quad \forall a,b\in\mathbb{C}.
%	\]
%	Its representation in an orthonormal basis is given by the matrix \(L\) with entries \(L_{ij} = \bra{v_i}\mathcal{L}\ket{v_j}\).
%	
%	\subsection*{Hermitian Operators}
%	
%	An operator \(\mathcal{H}\) is \emph{Hermitian} (or self-adjoint) if
%	\[
%	\bra{\phi}\mathcal{H}\ket{\psi} = \bra{\psi}\mathcal{H}\ket{\phi}^* \quad \forall \ket{\phi}, \ket{\psi} \in \mathcal{H}.
%	\]
%	Important properties include:
%	\begin{itemize}[leftmargin=2em]
%		\item All eigenvalues of a Hermitian operator are real.
%		\item Eigenvectors corresponding to distinct eigenvalues are orthogonal.
%	\end{itemize}
%	The eigenvalue equation is
%	\[
%	\mathcal{H}\ket{a} = a\ket{a}, \quad a \in \mathbb{R}.
%	\]
%	
%	\subsection*{Unitary Operators}
%	
%	An operator \(U\) is \emph{unitary} if
%	\[
%	U^\dagger U = U U^\dagger = I.
%	\]
%	Unitary operators preserve the inner product:
%	\[
%	\braket{U\psi}{U\phi} = \braket{\psi}{\phi},
%	\]
%	and their eigenvalues lie on the complex unit circle, i.e., if
%	\[
%	U\ket{u} = \lambda \ket{u},
%	\]
%	then \(|\lambda| = 1\).
%	
%	%%%%%%%%%%%%%%%%%%%%%%%%%%%%%%%%%%%%%%%%%%%%%%%%%%%%%%%%%%%%%%%%%%%%%%
%	\section{Eigenvalue Problems}
%	%%%%%%%%%%%%%%%%%%%%%%%%%%%%%%%%%%%%%%%%%%%%%%%%%%%%%%%%%%%%%%%%%%%%%%
%	
%	For a linear operator \(A\) on \(\mathcal{H}\), the eigenvalue problem is formulated as
%	\[
%	A\ket{a} = a\ket{a}, \quad a \in \mathbb{C},
%	\]
%	where \(\ket{a}\) is the eigenvector corresponding to the eigenvalue \(a\). In the context of Hermitian operators, the eigenvalues are guaranteed to be real, and the eigenvectors form a complete orthonormal set.
%	
%	%%%%%%%%%%%%%%%%%%%%%%%%%%%%%%%%%%%%%%%%%%%%%%%%%%%%%%%%%%%%%%%%%%%%%%
%	\section{Conclusion}
%	%%%%%%%%%%%%%%%%%%%%%%%%%%%%%%%%%%%%%%%%%%%%%%%%%%%%%%%%%%%%%%%%%%%%%%
%	
%	This document has outlined the foundational mathematical structures used in quantum mechanics. The rigorous, symbolic style—featuring Dirac’s bra-ket notation, matrix representations, and the properties of linear, Hermitian, and unitary operators—provides a framework suitable for advanced study and for publishing in a graduate-level text.
%	
%\end{document}

\documentclass[12pt]{report}
\usepackage[utf8]{inputenc}
\usepackage[T1]{fontenc}
\usepackage{amsmath, amssymb, amsfonts, amsthm, mathrsfs}
\usepackage{physics}       % Provides bra-ket commands and more
\usepackage{bm}            % Bold math symbols
\usepackage{geometry}      % Adjust page margins
\usepackage{hyperref}      % Hyperlinks in the PDF
\geometry{margin=1in}

% Custom macros for Dirac notation and common symbols
\renewcommand{\ket}[1]{\left| #1 \right\rangle}      % ket vector
\renewcommand{\bra}[1]{\left\langle #1 \right|}        % bra vector
\renewcommand{\braket}[2]{\left\langle #1 \middle| #2 \right\rangle} % inner product
\renewcommand{\norm}[1]{\left\lVert #1 \right\rVert}   % norm
\newcommand{\C}{\mathbb{C}}
\newcommand{\R}{\mathbb{R}}

% Theorem environments for rigorous statements
\newtheorem{theorem}{Theorem}[chapter]
\newtheorem{lemma}[theorem]{Lemma}
\newtheorem{definition}[theorem]{Definition}
\newtheorem{corollary}[theorem]{Corollary}
\newtheorem{remark}[theorem]{Remark}

\begin{document}
	
	%%%%%%%%%%%%%%%%%%%%%%%%%%%%%%%%%%%%%%%%%%%%%%%%%%%%%%%%%%%%%%%%%%%%%%
	\begin{titlepage}
		\centering
		\vspace*{2cm}
		{\Huge \bfseries Lecture Notes on Quantum Mechanics and Linear Algebra}\\[1.5cm]
		{\Large A Graduate-Level Mathematical Treatment}\\[2cm]
		{\Large Author Name}\\[1cm]
		{\Large \today}\\[2cm]
		\vfill
	\end{titlepage}
	%%%%%%%%%%%%%%%%%%%%%%%%%%%%%%%%%%%%%%%%%%%%%%%%%%%%%%%%%%%%%%%%%%%%%%
	
	\tableofcontents
	\newpage
	
	%%%%%%%%%%%%%%%%%%%%%%%%%%%%%%%%%%%%%%%%%%%%%%%%%%%%%%%%%%%%%%%%%%%%%%
	\chapter{Introduction}
	%%%%%%%%%%%%%%%%%%%%%%%%%%%%%%%%%%%%%%%%%%%%%%%%%%%%%%%%%%%%%%%%%%%%%%
	
	Quantum mechanics and linear algebra form the mathematical backbone of modern theoretical physics. In these notes we present a rigorous treatment of Hilbert spaces, Dirac's bra-ket notation, dual spaces, linear operators, and their spectral properties. This exposition is designed for advanced studies, where both mathematical precision and symbolic clarity are paramount.
	
	\chapter{Preliminaries}
	
	\section{Vector Spaces and Fields}
	\begin{definition}[Field]
		A \emph{field} \( \mathbb{F} \) is a set equipped with two binary operations (addition and multiplication) such that:
		\begin{enumerate}
			\item \( (\mathbb{F}, +) \) is an abelian group.
			\item \( (\mathbb{F}\setminus\{0\}, \cdot) \) is an abelian group.
			\item Distributivity holds: \( a\cdot(b+c) = a\cdot b + a\cdot c \) for all \( a,b,c\in\mathbb{F} \).
		\end{enumerate}
		Typical examples include \( \R \) and \( \C \).
	\end{definition}
	
	\begin{definition}[Vector Space]
		A \emph{vector space} \( V \) over a field \( \mathbb{F} \) is a set equipped with two operations (vector addition and scalar multiplication) that satisfy the usual axioms:
		\begin{itemize}
			\item Associativity and commutativity of addition.
			\item Existence of a zero vector \( \mathbf{0} \).
			\item Existence of additive inverses.
			\item Compatibility of scalar multiplication with field multiplication.
			\item Distributive properties.
		\end{itemize}
	\end{definition}
	
	\section{Normed and Inner Product Spaces}
	\begin{definition}[Normed Vector Space]
		A \emph{normed vector space} is a vector space \( V \) over \( \mathbb{F} \) equipped with a norm \( \norm{\cdot}: V \to \R \) satisfying:
		\begin{enumerate}
			\item \( \norm{v} \ge 0 \) for all \( v\in V \), and \( \norm{v}=0 \) if and only if \( v=0 \).
			\item \( \norm{\alpha v} = |\alpha|\, \norm{v} \) for all \( \alpha\in\mathbb{F} \) and \( v\in V \).
			\item Triangle inequality: \( \norm{u+v} \le \norm{u}+\norm{v} \) for all \( u,v\in V \).
		\end{enumerate}
	\end{definition}
	
	\begin{definition}[Inner Product Space]
		An \emph{inner product space} is a vector space \( V \) over \( \mathbb{F} \) together with an inner product 
		\[
		\langle \cdot, \cdot \rangle: V \times V \to \mathbb{F},
		\]
		satisfying for all \( u,v,w\in V \) and \( \alpha\in\mathbb{F} \):
		\begin{enumerate}
			\item \textbf{Conjugate symmetry:} \( \langle u, v \rangle = \overline{\langle v, u \rangle} \).
			\item \textbf{Linearity in the first argument:} \( \langle \alpha u + v, w \rangle = \alpha \langle u, w \rangle + \langle v, w \rangle \).
			\item \textbf{Positive-definiteness:} \( \langle v, v \rangle \ge 0 \), and \( \langle v, v \rangle = 0 \) if and only if \( v=0 \).
		\end{enumerate}
		The norm is induced by \( \norm{v} = \sqrt{\langle v, v \rangle} \).
	\end{definition}
	
	\section{Hilbert Spaces}
	\begin{definition}[Hilbert Space]
		A \emph{Hilbert space} \( \mathcal{H} \) is a complete inner product space, meaning that every Cauchy sequence in \( \mathcal{H} \) converges with respect to the norm induced by the inner product.
	\end{definition}
	
	\begin{remark}
		Any finite-dimensional inner product space is automatically complete and hence is a Hilbert space.
	\end{remark}
	
	\section{Dual Spaces and Dirac's Bra-Ket Notation}
	\begin{definition}[Dual Space]
		Let \( V \) be a vector space over \( \mathbb{F} \). The \emph{dual space} \( V^* \) is defined as
		\[
		V^* = \{ f: V \to \mathbb{F} \mid f \text{ is linear} \}.
		\]
	\end{definition}
	
	In quantum mechanics, vectors in a Hilbert space \( \mathcal{H} \) are denoted by \emph{kets} \( \ket{\psi} \) and the corresponding dual vectors by \emph{bras} \( \bra{\psi} \), with the inner product given by
	\[
	\braket{\phi}{\psi} = \bra{\phi}\ket{\psi}.
	\]
	
	\section{Basis, Linear Independence, and Orthonormal Sets}
	\begin{definition}[Linear Independence]
		A set \( \{v_i\}_{i \in I} \subset V \) is said to be \emph{linearly independent} if
		\[
		\sum_{i\in I} \alpha_i v_i = 0 \quad \Longrightarrow \quad \alpha_i = 0 \quad \text{for all } i,
		\]
		where only finitely many \( \alpha_i \) are nonzero.
	\end{definition}
	
	\begin{definition}[Basis]
		A set \( \{v_i\}_{i \in I} \) is a \emph{basis} of \( V \) if it is linearly independent and every \( v \in V \) can be written as a (finite or infinite) linear combination of the \( v_i \)'s.
	\end{definition}
	
	\begin{definition}[Orthonormal Basis]
		An \emph{orthonormal basis} for an inner product space \( V \) is a basis \( \{v_i\}_{i \in I} \) such that
		\[
		\langle v_i, v_j \rangle = \delta_{ij},
		\]
		with \( \delta_{ij} \) being the Kronecker delta.
	\end{definition}
	
	The completeness relation in a Hilbert space \( \mathcal{H} \) with orthonormal basis \( \{ \ket{v_i} \} \) is given by
	\[
	\sum_{i\in I} \ket{v_i}\bra{v_i} = I,
	\]
	where \( I \) is the identity operator on \( \mathcal{H} \).
	
	\section{Linear Operators and Their Matrix Representations}
	\begin{definition}[Linear Operator]
		A mapping \( \mathcal{L}: \mathcal{H} \to \mathcal{H} \) is called a \emph{linear operator} if for all \( \ket{\psi}, \ket{\phi} \in \mathcal{H} \) and \( \alpha,\beta \in \mathbb{F} \),
		\[
		\mathcal{L}(\alpha \ket{\psi} + \beta \ket{\phi}) = \alpha\,\mathcal{L}\ket{\psi} + \beta\,\mathcal{L}\ket{\phi}.
		\]
	\end{definition}
	
	Given an orthonormal basis \( \{ \ket{v_i} \} \) of \( \mathcal{H} \), the matrix representation of \( \mathcal{L} \) is
	\[
	L_{ij} = \bra{v_i}\mathcal{L}\ket{v_j}.
	\]
	
	\begin{definition}[Change of Basis]
		Let \( \{ \ket{v_i} \} \) and \( \{ \ket{w_i} \} \) be two orthonormal bases of \( \mathcal{H} \). There exists a unitary operator \( U \) such that
		\[
		\ket{w_i} = U\ket{v_i}.
		\]
		Under this transformation, the representation of an operator \( \mathcal{L} \) changes as
		\[
		L' = U^\dagger L\,U.
		\]
	\end{definition}
	
	\section{Hermitian and Unitary Operators}
	\begin{definition}[Hermitian Operator]
		A linear operator \( \mathcal{H}:\mathcal{H}\to\mathcal{H} \) is \emph{Hermitian} (or self-adjoint) if
		\[
		\bra{\psi}\mathcal{H}\ket{\phi} = \overline{\bra{\phi}\mathcal{H}\ket{\psi}}
		\]
		for all \( \ket{\psi}, \ket{\phi}\in\mathcal{H} \). Equivalently, \( \mathcal{H} = \mathcal{H}^\dagger \).
	\end{definition}
	
	\begin{theorem}[Properties of Hermitian Operators]
		If \( \mathcal{H} \) is Hermitian, then:
		\begin{enumerate}
			\item All eigenvalues of \( \mathcal{H} \) are real.
			\item Eigenvectors corresponding to distinct eigenvalues are orthogonal.
		\end{enumerate}
	\end{theorem}
	
	\begin{definition}[Unitary Operator]
		A linear operator \( U: \mathcal{H} \to \mathcal{H} \) is \emph{unitary} if
		\[
		U^\dagger U = U U^\dagger = I.
		\]
	\end{definition}
	
	\begin{remark}
		Unitary operators preserve inner products:
		\[
		\braket{U\psi}{U\phi} = \braket{\psi}{\phi} \quad \text{for all } \ket{\psi},\ket{\phi}\in\mathcal{H}.
		\]
	\end{remark}
	
	\section{Eigenvalues, Eigenvectors, and the Spectral Theorem}
	\begin{definition}[Eigenvalue and Eigenvector]
		Let \( \mathcal{L} \) be a linear operator on \( \mathcal{H} \). A scalar \( \lambda \in \mathbb{F} \) is an \emph{eigenvalue} of \( \mathcal{L} \) if there exists a nonzero vector \( \ket{\psi} \) such that
		\[
		\mathcal{L}\ket{\psi} = \lambda \ket{\psi}.
		\]
		The vector \( \ket{\psi} \) is called an \emph{eigenvector} corresponding to \( \lambda \).
	\end{definition}
	
	\begin{theorem}[Spectral Theorem for Finite-Dimensional Hermitian Operators]
		Let \( \mathcal{H} \) be a finite-dimensional Hilbert space and let \( \mathcal{A} \) be a Hermitian operator on \( \mathcal{H} \). Then:
		\begin{enumerate}
			\item All eigenvalues of \( \mathcal{A} \) are real.
			\item There exists an orthonormal basis of \( \mathcal{H} \) consisting of eigenvectors of \( \mathcal{A} \).
			\item \( \mathcal{A} \) can be expressed in its spectral decomposition:
			\[
			\mathcal{A} = \sum_{i} \lambda_i \ket{v_i}\bra{v_i},
			\]
			where \( \lambda_i \) are the eigenvalues and \( \{ \ket{v_i} \} \) is an orthonormal basis.
		\end{enumerate}
	\end{theorem}
	
	\begin{remark}
		In quantum mechanics, the spectral theorem underpins the representation of observables by Hermitian operators, ensuring that measurable quantities are real and that the system can be described by a complete set of eigenstates.
	\end{remark}
	
	%%%%%%%%%%%%%%%%%%%%%%%%%%%%%%%%%%%%%%%%%%%%%%%%%%%%%%%%%%%%%%%%%%%%%%
	\chapter{Hilbert Spaces and Inner Product Spaces}
	%%%%%%%%%%%%%%%%%%%%%%%%%%%%%%%%%%%%%%%%%%%%%%%%%%%%%%%%%%%%%%%%%%%%%%
	
	\section{Vector Spaces, Norms, and Inner Products}
	Let \( V \) be a vector space over the field \( \C \) (or \( \R \)). An \emph{inner product} on \( V \) is a function 
	\[
	\langle \cdot , \cdot \rangle : V \times V \to \C,
	\]
	satisfying, for all \(\mathbf{u}, \mathbf{v}, \mathbf{w} \in V\) and all \(\alpha \in \C\):
	\begin{enumerate}
		\item \textbf{Conjugate Symmetry:} \(\langle \mathbf{u}, \mathbf{v} \rangle = \overline{\langle \mathbf{v}, \mathbf{u} \rangle}\).
		\item \textbf{Linearity (in the first argument):} \(\langle \alpha \mathbf{u} + \mathbf{v}, \mathbf{w} \rangle = \alpha \langle \mathbf{u}, \mathbf{w} \rangle + \langle \mathbf{v}, \mathbf{w} \rangle\).
		\item \textbf{Positive-Definiteness:} \(\langle \mathbf{v}, \mathbf{v} \rangle \ge 0\), with equality if and only if \(\mathbf{v} = \mathbf{0}\).
	\end{enumerate}
	The induced norm is defined by
	\[
	\norm{\mathbf{v}} = \sqrt{\langle \mathbf{v}, \mathbf{v} \rangle}.
	\]
	
	\section{Hilbert Spaces}
	\begin{definition}[Hilbert Space]
		A \emph{Hilbert space} \( \mathcal{H} \) is a complete inner product space. That is, every Cauchy sequence \(\{\ket{\psi_n}\}\) in \(\mathcal{H}\) converges (with respect to the norm \(\norm{\cdot}\)) to a vector in \(\mathcal{H}\).
	\end{definition}
	
	\section{Dual Spaces and Dirac’s Bra-Ket Notation}
	\begin{definition}[Dual Space]
		Let \( V \) be a vector space over \( \C \). The \emph{dual space} \( V^* \) is defined as the set of all linear functionals 
		\[
		f: V \to \C,
		\]
		that is, for all \(\mathbf{u}, \mathbf{v} \in V\) and all \(\alpha, \beta \in \C\),
		\[
		f(\alpha \mathbf{u} + \beta \mathbf{v}) = \alpha f(\mathbf{u}) + \beta f(\mathbf{v}).
		\]
	\end{definition}
	
	In quantum mechanics, elements of \( \mathcal{H} \) are denoted by \emph{kets} \( \ket{\psi} \), while the corresponding dual vectors (elements of \( \mathcal{H}^* \)) are denoted by \emph{bras} \( \bra{\psi} \). The inner product is then written as
	\[
	\braket{\phi}{\psi} = \bra{\phi}\ket{\psi}.
	\]
	
	%%%%%%%%%%%%%%%%%%%%%%%%%%%%%%%%%%%%%%%%%%%%%%%%%%%%%%%%%%%%%%%%%%%%%%
	\chapter{Orthonormal Bases and Completeness}
	%%%%%%%%%%%%%%%%%%%%%%%%%%%%%%%%%%%%%%%%%%%%%%%%%%%%%%%%%%%%%%%%%%%%%%
	
	\section{Basis and Linear Independence}
	A set of vectors \(\{\ket{v_1}, \ket{v_2}, \dots, \ket{v_n}\}\) in \(\mathcal{H}\) is said to be \emph{linearly independent} if
	\[
	\sum_{i=1}^{n} \alpha_i \ket{v_i} = \mathbf{0} \quad \Longrightarrow \quad \alpha_i = 0 \quad \text{for all } i.
	\]
	If every vector \(\ket{\psi} \in \mathcal{H}\) can be expressed as a linear combination of \(\{\ket{v_i}\}\),
	\[
	\ket{\psi} = \sum_{i=1}^{n} \psi_i \ket{v_i},
	\]
	then \(\{\ket{v_i}\}\) is a \emph{basis} of \(\mathcal{H}\).
	
	\section{Orthonormal Bases}
	\begin{definition}[Orthonormal Basis]
		A set \(\{\ket{v_i}\}_{i\in I}\) in \(\mathcal{H}\) is an \emph{orthonormal set} if
		\[
		\braket{v_i}{v_j} = \delta_{ij} \quad \text{for all } i,j \in I,
		\]
		where \(\delta_{ij}\) is the Kronecker delta. If the linear span of \(\{\ket{v_i}\}\) is dense in \(\mathcal{H}\), then it is called an \emph{orthonormal basis}.
	\end{definition}
	
	The completeness relation is given by
	\[
	\sum_{i\in I} \ket{v_i}\bra{v_i} = I,
	\]
	where \( I \) is the identity operator on \(\mathcal{H}\).
	
	%%%%%%%%%%%%%%%%%%%%%%%%%%%%%%%%%%%%%%%%%%%%%%%%%%%%%%%%%%%%%%%%%%%%%%
	\chapter{Linear Operators and Their Representations}
	%%%%%%%%%%%%%%%%%%%%%%%%%%%%%%%%%%%%%%%%%%%%%%%%%%%%%%%%%%%%%%%%%%%%%%
	
	\section{Linear Operators}
	\begin{definition}[Linear Operator]
		Let \(\mathcal{H}\) be a Hilbert space. A mapping \(\mathcal{L}:\mathcal{H} \to \mathcal{H}\) is a \emph{linear operator} if for all \(\ket{\phi}, \ket{\psi} \in \mathcal{H}\) and \(\alpha, \beta \in \C\),
		\[
		\mathcal{L}(\alpha\ket{\phi} + \beta\ket{\psi}) = \alpha\,\mathcal{L}\ket{\phi} + \beta\,\mathcal{L}\ket{\psi}.
		\]
	\end{definition}
	
	Given an orthonormal basis \(\{\ket{v_i}\}\) of \(\mathcal{H}\), the operator \(\mathcal{L}\) can be represented by the matrix \( [L_{ij}] \) with entries
	\[
	L_{ij} = \bra{v_i}\mathcal{L}\ket{v_j}.
	\]
	
	\section{Change of Basis and Similarity Transformations}
	Let \(\{\ket{v_i}\}\) and \(\{\ket{w_i}\}\) be two orthonormal bases of \(\mathcal{H}\). There exists a unitary operator \( U \) such that
	\[
	\ket{w_i} = U\ket{v_i}.
	\]
	Under this change of basis, the matrix representation \( L' \) of the operator \( \mathcal{L} \) is given by
	\[
	L' = U^\dagger L\,U.
	\]
	This is known as a \emph{similarity transformation}.
	
	%%%%%%%%%%%%%%%%%%%%%%%%%%%%%%%%%%%%%%%%%%%%%%%%%%%%%%%%%%%%%%%%%%%%%%
	\chapter{Hermitian and Unitary Operators}
	%%%%%%%%%%%%%%%%%%%%%%%%%%%%%%%%%%%%%%%%%%%%%%%%%%%%%%%%%%%%%%%%%%%%%%
	
	\section{Hermitian Operators}
	\begin{definition}[Hermitian Operator]
		An operator \(\mathcal{H}:\mathcal{H}\to\mathcal{H}\) is called \emph{Hermitian} (or self-adjoint) if
		\[
		\bra{\phi}\mathcal{H}\ket{\psi} = \overline{\bra{\psi}\mathcal{H}\ket{\phi}} \quad \forall \ket{\phi},\ket{\psi}\in\mathcal{H}.
		\]
	\end{definition}
	
	Important properties include:
	\begin{itemize}
		\item The eigenvalues of a Hermitian operator are real.
		\item Eigenvectors corresponding to distinct eigenvalues are orthogonal.
	\end{itemize}
	
	\section{Unitary Operators}
	\begin{definition}[Unitary Operator]
		An operator \( U:\mathcal{H}\to\mathcal{H} \) is \emph{unitary} if
		\[
		U^\dagger U = U U^\dagger = I,
		\]
		where \( U^\dagger \) denotes the Hermitian (conjugate) transpose of \( U \), and \( I \) is the identity operator.
	\end{definition}
	
	Unitary operators preserve the inner product:
	\[
	\braket{U\psi}{U\phi} = \braket{\psi}{\phi} \quad \forall \ket{\psi},\ket{\phi}\in\mathcal{H}.
	\]
	Consequently, they are norm-preserving and invertible, with \( U^{-1} = U^\dagger \).
	
	%%%%%%%%%%%%%%%%%%%%%%%%%%%%%%%%%%%%%%%%%%%%%%%%%%%%%%%%%%%%%%%%%%%%%%
	\chapter{Eigenvalue Problems and Spectral Theory}
	%%%%%%%%%%%%%%%%%%%%%%%%%%%%%%%%%%%%%%%%%%%%%%%%%%%%%%%%%%%%%%%%%%%%%%
	
	\section{Eigenvalue Equations}
	Let \( \mathcal{L} \) be a linear operator on \(\mathcal{H}\). An \emph{eigenvalue} \(\lambda\) and a corresponding non-zero eigenvector \(\ket{\psi}\) satisfy
	\[
	\mathcal{L}\ket{\psi} = \lambda \ket{\psi}.
	\]
	
	\section{Spectral Decomposition}
	For a Hermitian operator \( \mathcal{H} \) with a complete orthonormal set of eigenvectors \(\{\ket{v_i}\}\) and corresponding eigenvalues \(\lambda_i\), the spectral theorem guarantees that
	\[
	\mathcal{H} = \sum_{i} \lambda_i \ket{v_i}\bra{v_i}.
	\]
	For unitary operators, every eigenvalue \(\lambda\) satisfies \(|\lambda|=1\).
	
	%%%%%%%%%%%%%%%%%%%%%%%%%%%%%%%%%%%%%%%%%%%%%%%%%%%%%%%%%%%%%%%%%%%%%%
	\chapter{Conclusion}
	%%%%%%%%%%%%%%%%%%%%%%%%%%%%%%%%%%%%%%%%%%%%%%%%%%%%%%%%%%%%%%%%%%%%%%
	
	In these notes, we have constructed a detailed, mathematically rigorous framework for quantum mechanics based on Hilbert space theory and linear algebra. Starting from the fundamental definitions of vector spaces, inner products, and dual spaces, we have introduced Dirac's bra-ket notation, developed the theory of linear operators, and examined the special classes of Hermitian and unitary operators along with their spectral properties. This formalism underpins many modern theoretical developments and provides a solid foundation for further studies in both mathematics and physics.
	
	
	%%%%%%%%%%%%%%%%%%%%%%%%%%%%%%%%%%%%%%%%%%%%%%%%%%%%%%%%%%%%%%%%%%%%%%
	\chapter{Spectral Theory of Hermitian Operators}
	%%%%%%%%%%%%%%%%%%%%%%%%%%%%%%%%%%%%%%%%%%%%%%%%%%%%%%%%%%%%%%%%%%%%%%
	
	\section{The Spectral Theorem}
	\begin{theorem}[Spectral Theorem for Hermitian Operators]
		Let \(\mathcal{H}\) be a finite-dimensional Hilbert space and let \(\mathcal{A}:\mathcal{H}\to\mathcal{H}\) be a Hermitian operator, i.e., \(\mathcal{A} = \mathcal{A}^\dagger\). Then:
		\begin{enumerate}
			\item All eigenvalues of \(\mathcal{A}\) are real.
			\item There exists an orthonormal basis of \(\mathcal{H}\) consisting of eigenvectors of \(\mathcal{A}\).
		\end{enumerate}
	\end{theorem}
	
	\begin{proof}
		We provide a detailed proof by following these steps:
		
		\textbf{Step 1. Eigenvalues are Real:}  
		Let \(\ket{\psi}\) be an eigenvector of \(\mathcal{A}\) corresponding to the eigenvalue \(\lambda\), so that
		\[
		\mathcal{A}\ket{\psi} = \lambda \ket{\psi}.
		\]
		Taking the inner product with \(\ket{\psi}\) on the left yields
		\[
		\bra{\psi}\mathcal{A}\ket{\psi} = \lambda \braket{\psi}{\psi}.
		\]
		Since \(\mathcal{A}\) is Hermitian, we have
		\[
		\bra{\psi}\mathcal{A}\ket{\psi} = \overline{\bra{\psi}\mathcal{A}\ket{\psi}}.
		\]
		Thus, \(\lambda \braket{\psi}{\psi} = \overline{\lambda}\braket{\psi}{\psi}\). Because \(\braket{\psi}{\psi} > 0\) for \(\ket{\psi} \neq 0\), it follows that
		\[
		\lambda = \overline{\lambda},
		\]
		so \(\lambda\) is real.
		
		\textbf{Step 2. Orthogonality of Eigenvectors:}  
		Let \(\ket{\psi}\) and \(\ket{\phi}\) be eigenvectors corresponding to distinct eigenvalues \(\lambda\) and \(\mu\), respectively, with \(\lambda \neq \mu\). Then,
		\[
		\mathcal{A}\ket{\psi} = \lambda \ket{\psi} \quad \text{and} \quad \mathcal{A}\ket{\phi} = \mu \ket{\phi}.
		\]
		Taking the inner product \(\bra{\phi}\) with the first equation gives
		\[
		\bra{\phi}\mathcal{A}\ket{\psi} = \lambda \braket{\phi}{\psi}.
		\]
		On the other hand, by using the Hermitian property,
		\[
		\bra{\phi}\mathcal{A}\ket{\psi} = \overline{\bra{\psi}\mathcal{A}\ket{\phi}} = \overline{\mu \braket{\psi}{\phi}} = \mu \braket{\phi}{\psi},
		\]
		since \(\mu\) is real. Equating the two expressions, we obtain
		\[
		\lambda \braket{\phi}{\psi} = \mu \braket{\phi}{\psi}.
		\]
		Because \(\lambda \neq \mu\), it must be that \(\braket{\phi}{\psi} = 0\), showing that the eigenvectors are orthogonal.
		
		\textbf{Step 3. Completeness:}  
		Since \(\mathcal{H}\) is finite-dimensional, the operator \(\mathcal{A}\) has a full set of eigenvalues (counted with multiplicity). One can show that the sum of the dimensions of the eigenspaces equals \(\dim(\mathcal{H})\). By applying the Gram-Schmidt process if necessary, we can obtain an orthonormal basis for each eigenspace. Combining these orthonormal sets, we get an orthonormal basis for \(\mathcal{H}\) consisting entirely of eigenvectors of \(\mathcal{A}\).
		
		\textbf{Conclusion:}  
		Thus, we have demonstrated that all eigenvalues of \(\mathcal{A}\) are real and that \(\mathcal{H}\) admits an orthonormal basis of eigenvectors. This completes the proof.
	\end{proof}
	
	\begin{remark}
		The spectral theorem has profound implications in quantum mechanics, where observables are represented by Hermitian operators and the eigenvectors correspond to measurable states.
	\end{remark}
	
	%%%%%%%%%%%%%%%%%%%%%%%%%%%%%%%%%%%%%%%%%%%%%%%%%%%%%%%%%%%%%%%%%%%%%%
	\chapter{Additional Detailed Reasoning in Graduate-Level Notes}
	%%%%%%%%%%%%%%%%%%%%%%%%%%%%%%%%%%%%%%%%%%%%%%%%%%%%%%%%%%%%%%%%%%%%%%
	
	In graduate-level texts, you may also include:
	\begin{itemize}
		\item \textbf{Motivating examples:} For instance, explicit diagonalization of simple Hermitian matrices.
		\item \textbf{Historical context:} A brief discussion of how the spectral theorem evolved in mathematics and physics.
		\item \textbf{Connections with other theories:} Detailed reasoning on how these concepts extend to infinite-dimensional spaces, distribution theory, and functional analysis.
		\item \textbf{Exercises and proofs:} Additional exercises with complete solutions to deepen understanding.
	\end{itemize}
	
	The above snippet illustrates one way to incorporate extensive, detailed reasoning into your lecture notes. Each theorem is stated formally and followed by a rigorous proof that explains every step in detail. This style is well suited for a graduate-level publication where mathematical rigor is paramount.
	
	
\end{document}
