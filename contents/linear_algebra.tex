\chapter{Linear Operators on Finite-Dimensional Hilbert Spaces}

Quantum computing relies fundamentally on the language of linear algebra. Quantum states are vectors in complex Hilbert spaces, and quantum operations are linear operators acting on these spaces.

\section{Vector Spaces and Dirac's Bra--Ket Notation}
\begin{definition}[Hilbert Space]
	A \emph{Hilbert space} $\mathcal{H}$ is a complete inner product space over the field $\mathbb{F}$ (either $\mathbb{R}$ or $\mathbb{C}$). Concretely, $\mathcal{H}$ satisfies:
	\begin{enumerate}
		\item (Vector Space) $\mathcal{H}$ is a vector space over $\mathbb{F}$.
		\item (Inner Product) There exists a map
		\[
		\langle \cdot,\cdot\rangle:\mathcal{H}\times\mathcal{H}\to\mathbb{F}
		\]
		satisfying for all $x,y,z\in\mathcal{H}$ and $\alpha,\beta\in\mathbb{F}$:
		\begin{enumerate}
			\item (Conjugate Symmetry) $\langle x,y\rangle=\overline{\langle y,x\rangle}$.
			\item (Linearity) $\langle \alpha x+\beta y,\,z\rangle=\alpha\langle x,z\rangle+\beta\langle y,z\rangle$.
			\item (Positive-Definiteness) $\langle x,x\rangle\ge0$, and $\langle x,x\rangle=0$ if and only if $x=0$.
		\end{enumerate}
		\item (Norm) The norm induced by the inner product, $\|x\|=\sqrt{\langle x,x\rangle}$, defines a metric $d(x,y)=\|x-y\|$.
		\item (Completeness) $\mathcal{H}$ is complete with respect to the metric $d$, i.e., every Cauchy sequence in $\mathcal{H}$ converges to a limit in $\mathcal{H}$.
	\end{enumerate}
\end{definition}

%\begin{example}
%	The space \[
%	\ell^2(\mathbb{N})=\set{(x_n):\sum\abs{x_n}^2<\infty}
%	\] with inner product $\langle x,y\rangle=\sum_{n=1}^\infty x_n\overline{y_n}$ is a Hilbert space.
%\end{example}

\begin{definition}[Ket]
	Let $\mathcal{H}$ be a Hilbert space over $\mathbb{C}$. A \emph{ket} $\ket{\psi}$ denotes an element $\psi\in\mathcal{H}$.
\end{definition}

\begin{definition}[Bra]
	To each ket $\ket{\psi}\in\mathcal{H}$, there corresponds a unique continuous linear functional (via the Riesz representation theorem) denoted by the \emph{bra} $\bra{\psi}\colon\mathcal{H}\to\mathbb{C}$, defined by
	\[\bra{\psi}\left(\ket{\phi}\right)=\langle\psi,\phi\rangle,\quad\forall\ket{\phi}\in\mathcal{H}.\]
\end{definition}
\begin{remark}
The mapping $\ket{\psi}\mapsto\bra{\psi}$ is an antilinear isometric isomorphism $J:\mathcal{H}\to\mathcal{H}^*$, where $\mathcal{H}^*$ is the dual space.
\end{remark}

\newpage
%\begin{definition}[Inner Product]
%	The inner product of two kets $\ket{\phi},\ket{\psi}\in\mathcal{H}$ is denoted as
%	\[\bra{\phi}\cdot\ket{\psi}=\braket{\phi}{\psi}
%%	\braket{\phi|\psi}=\bra{\phi}(\ket{\psi})=\langle\phi,\psi\rangle.\]
%\]
%	This quantity satisfies:
%	\begin{enumerate}
%		\item $\braket{\phi|\psi}=\overline{\braket{\psi|\phi}}$.
%		\item $\braket{\alpha\phi+\beta\chi|\psi}=\alpha^*\braket{\phi|\psi}+\beta^*\braket{\chi|\psi}$.
%		\item $\braket{\phi|\phi}\ge0$, with equality iff $\ket{\phi}=0$.
%	\end{enumerate}
%\end{definition}
%
%\begin{proposition}[Outer Product]
%	For $\ket{\phi},\ket{\psi}\in\mathcal{H}$, the \emph{outer product} \(\ket{\phi}\bra{\psi}\) denotes the rank-one operator on $\mathcal{H}$ given by
%	\[(\ket{\phi}\bra{\psi})\ket{\chi}=\ket{\phi}\braket{\psi|\chi},\quad \forall\ket{\chi}\in\mathcal{H}.
%	\]
%	In matrix terms, if $\ket{\phi}$ and $\ket{\psi}$ have components $\phi_i,\psi_j$ in an orthonormal basis, then $(\ket{\phi}\bra{\psi})_{ij}=\phi_i\overline{\psi_j}$.
%\end{proposition}
%
%\begin{example}
%	In $\mathbb{C}^2$ with standard basis $\{\ket{0},\ket{1}\}$, the outer product $\ket{0}\bra{1}$ is the matrix
%	\[\begin{pmatrix}1\\0\end{pmatrix}(0\;1)=\begin{pmatrix}0 & 1\\ 0 & 0\end{pmatrix}.
%	\]
%\end{example}


\newpage
%\begin{definition}[Complex Vector Space]
%	A \emph{complex vector space} $\mathcal V$ is a set with operations of addition and scalar multiplication (by $\mathbb C$) satisfying the axioms of a vector space.
%\end{definition}
%
%In quantum mechanics, we denote vectors using \emph{Dirac notation}:
%\begin{itemize}
%	\item A vector $\phi\in\mathcal V$ is written as a \emph{ket} $\ket{\phi}$.
%	\item Its dual (conjugate transpose) is written as a \emph{bra} $\bra{\phi}$.
%\end{itemize}
%The inner product between $\ket{\phi}$ and $\ket{\psi}$ is $\braket{\phi|\psi}\in\mathbb C$.

%\section{Inner Product and Norm}
%\begin{definition}[Inner Product]
%	An inner product on $\mathcal V$ is a map $\braket{\cdot|\cdot}:\mathcal V\times\mathcal V\to\mathbb C$ satisfying conjugate symmetry, linearity in the second argument, and positive-definiteness.
%\end{definition}
%
%\begin{definition}[Norm]
%	The norm of $\ket{\psi}$ is $\|\psi\|=\sqrt{\braket{\psi|\psi}}$. A vector of unit norm is called a \emph{normalized} state.
%\end{definition}

%\begin{definition}[Orthogonality and Orthonormality]
%	Two vectors $\ket{\phi},\ket{\psi}$ are \emph{orthogonal} if $\braket{\phi|\psi}=0$. A set is \emph{orthonormal} if each vector has unit norm and they are pairwise orthogonal.
%\end{definition}

%\section{Bases and Dimension}
%\begin{definition}[Basis]
%	An orthonormal basis of $\mathcal V$ is a set $\{\ket{v_i}\}_{i=1}^n$ such that any $\ket{\psi}\in\mathcal V$ can be uniquely expressed as $\ket{\psi}=\sum_i c_i\ket{v_i}$, $c_i=\braket{v_i|\psi}$.
%\end{definition}
%The dimension $\dim\mathcal V=n$ is the number of basis vectors.
%
%\section{Linear Operators}
%\begin{definition}[Linear Operator]
%	A map $L: \mathcal V\to\mathcal V$ is \emph{linear} if $L(\alpha\ket{\phi}+\beta\ket{\psi})=\alpha L\ket{\phi}+\beta L\ket{\psi}$ for all scalars $\alpha,\beta\in\mathbb C$.
%\end{definition}
%
%Given an orthonormal basis $\{\ket{v_i}\}$, any operator $L$ has a matrix representation $L_{ij}=\bra{v_i}L\ket{v_j}$. Acting on a state $\ket{\psi}$ with components $\psi_j$, we get $\ket{\phi}=L\ket{\psi}$ with $\phi_i=\sum_j L_{ij}\psi_j$.

\subsection{Adjoint, Hermitian, and Unitary Operators}
\begin{definition}[Adjoint]
	The \emph{adjoint} $L^\dagger$ satisfies $\bra{\phi}L\ket{\psi}=\bra{L^\dagger\phi}\psi$ for all states.
\end{definition}

\begin{definition}[Hermitian Operator]
	An operator $H$ is \emph{Hermitian} if $H=H^\dagger$. Its eigenvalues are real, and eigenvectors corresponding to distinct eigenvalues are orthogonal.
\end{definition}

\begin{definition}[Unitary Operator]
	An operator $U$ is \emph{unitary} if $U^\dagger U=I$. Unitary operators preserve norms and inner products.
\end{definition}

\section{Eigenvalues and Eigenvectors}
\begin{definition}[Eigenvalue Equation]
	For $L$ a linear operator, an eigenvector $\ket{v}$ and eigenvalue $\lambda$ satisfy
	\[L\ket{v}=\lambda\ket{v}.\]
\end{definition}
Spectral decomposition: any Hermitian $H$ can be written as $H=\sum_i h_i\ket{h_i}\bra{h_i}$.

\section{Quantum States and Observables}
A quantum state is represented by a normalized vector $\ket{\psi}$. Observables correspond to Hermitian operators; measurement yields an eigenvalue with probability $|\braket{h_i|\psi}|^2$.

\section{Quantum Gates as Unitary Transformations}
Elementary gates act on qubits ($2$-dimensional Hilbert spaces):
\begin{align*}
	X &= \begin{pmatrix}0 & 1\\1 & 0\end{pmatrix}, & Y &= \begin{pmatrix}0 & -i\\ i & 0\end{pmatrix}, & Z &= \begin{pmatrix}1 & 0\\ 0 & -1\end{pmatrix}, \\
	H &= \frac{1}{\sqrt2}\begin{pmatrix}1 & 1\\1 & -1\end{pmatrix}.
\end{align*}
These are unitary and represent rotation and superposition operations.
\begin{definition}[Standard and Hadamard Quantum States]
	Let $\mathcal H=\mathbb C^2$ be the single‐qubit Hilbert space with the canonical orthonormal basis
	\[
	\ket{0}=\begin{pmatrix}1\\0\end{pmatrix},
	\qquad
	\ket{1}=\begin{pmatrix}0\\1\end{pmatrix}.
	\]
	We then define the \emph{Hadamard basis} states by applying the Hadamard operator
	\[
	H=\frac1{\sqrt2}\begin{pmatrix}1 & 1\\1 & -1\end{pmatrix}
	\]
	to the computational basis as follows:
	\[
	\ket{+}\;=\;H\ket{0}
	\;=\;\frac1{\sqrt2}\bigl(\ket{0}+\ket{1}\bigr)
	=\frac1{\sqrt2}\begin{pmatrix}1\\1\end{pmatrix},
	\]
	\[
	\ket{-}\;=\;H\ket{1}
	\;=\;\frac1{\sqrt2}\bigl(\ket{0}-\ket{1}\bigr)
	=\frac1{\sqrt2}\begin{pmatrix}1\\-1\end{pmatrix}.
	\]
	These four vectors satisfy the following orthonormality relations:
	\[
	\langle 0 \mid 0\rangle = \langle 1 \mid 1\rangle = \langle + \mid +\rangle = \langle - \mid -\rangle = 1,
	\]
	\[
	\langle 0 \mid 1\rangle = \langle + \mid -\rangle = 0,
	\]
	and in fact
	\[
	\langle 0 \mid +\rangle = \langle 0 \mid -\rangle = \langle 1 \mid +\rangle = \langle 1 \mid -\rangle = \tfrac1{\sqrt2}(\pm1),
	\]
	so that each of $\ket0,\ket1,\ket+,\ket-$ has unit norm:
	\[
	\|\ket\psi\| \;=\;\sqrt{\langle\psi|\psi\rangle} \;=\;1
	\quad
	\text{for }
	\ket\psi\in\{\ket0,\ket1,\ket+,\ket-\}.
	\]
\end{definition}
\[
\ket{0}=\begin{pmatrix}1\\0\end{pmatrix},\quad
\ket{1}=\begin{pmatrix}0\\1\end{pmatrix},\quad
\ket{+}=\frac1{\sqrt2}\begin{pmatrix}1\\1\end{pmatrix},\quad
\ket{-}=\frac1{\sqrt2}\begin{pmatrix}1\\-1\end{pmatrix}.
\]
\textbf{Norms:}
\[
\langle0|0\rangle
= \begin{pmatrix}1&0\end{pmatrix}\begin{pmatrix}1\\0\end{pmatrix}
=1,
\quad
\langle1|1\rangle
=(0\;1)\begin{pmatrix}0\\1\end{pmatrix}
=1,
\]
\[
\langle+|+\rangle
=\frac1{2}(1\;1)\begin{pmatrix}1\\1\end{pmatrix}
=\frac1{2}(1+1)=1,
\quad
\langle-|-\,\rangle
=\frac1{2}(1\;-1)\begin{pmatrix}1\\-1\end{pmatrix}
=\frac1{2}(1+1)=1.
\]
\newpage\noindent
\textbf{Computational–Hadamard overlaps:}
\[
\langle0|+\rangle
=(1\;0)\frac1{\sqrt2}\begin{pmatrix}1\\1\end{pmatrix}
=\frac1{\sqrt2}(1\cdot1+0\cdot1)
=\frac1{\sqrt2},
\]
\[
\langle0|-\rangle
=(1\;0)\frac1{\sqrt2}\begin{pmatrix}1\\-1\end{pmatrix}
=\frac1{\sqrt2}(1\cdot1+0\cdot(-1))
=\frac1{\sqrt2},
\]
\[
\langle1|+\rangle
=(0\;1)\frac1{\sqrt2}\begin{pmatrix}1\\1\end{pmatrix}
=\frac1{\sqrt2}(0\cdot1+1\cdot1)
=\frac1{\sqrt2},
\]
\[
\langle1|-\rangle
=(0\;1)\frac1{\sqrt2}\begin{pmatrix}1\\-1\end{pmatrix}
=\frac1{\sqrt2}(0\cdot1+1\cdot(-1))
=-\frac1{\sqrt2}.
\]
\textbf{Computational–computational and Hadamard–Hadamard orthogonality:}
\[
\langle0|1\rangle
=(1\;0)\begin{pmatrix}0\\1\end{pmatrix}
=0,
\quad
\langle+|-\rangle
=\frac1{2}(1\;1)\begin{pmatrix}1\\-1\end{pmatrix}
=\frac1{2}(1-1)=0.
\]
All other overlaps follow by conjugation or symmetry.  Thus the set 
\(\{\ket0,\ket1,\ket+,\ket-\}\) 
is orthonormal and each has unit norm. 
%\section{Tensor Products and Composite Systems}
%For composite systems $\mathcal V_A\otimes\mathcal V_B$, basis states are tensor products $\ket{v_i}\otimes\ket{w_j}$. A two-qubit state is a vector in $\mathbb C^2\otimes\mathbb C^2=\mathbb C^4$.

%\section{Conclusion}
%Linear algebra provides the mathematical framework for quantum computing. Mastery of vector spaces, operators, and their spectral properties is essential to understanding quantum algorithms.

% ---------------------------------------------------------------------------------------
%\begin{definition}[Adjoint Operator]
%	The \emph{adjoint} $L^\dagger$ of a linear operator $L$ is defined by
%	\[ \braket{\phi | L\psi} = \braket{L^\dagger\phi | \psi}, \quad \forall\ket{\phi},\ket{\psi}\in\mathcal H. \]
%\end{definition}
%
%\begin{definition}[Hermitian Operator]
%	An operator $H$ is \emph{Hermitian} if $H=H^\dagger$. Its spectrum lies on the real axis, and it admits a spectral decomposition $H=\sum_j h_j\ket{h_j}\bra{h_j}$ with orthonormal eigenvectors $\{\ket{h_j}\}$.
%\end{definition}
%
%\begin{definition}[Unitary Operator]
%	An operator $U$ is \emph{unitary} if $U^\dagger U = U U^\dagger = I$, equivalently $\|U\psi\|=\|\psi\|$ for all $\ket{\psi}$.
%\end{definition}

\section{Eigenvalue Problems}
\begin{definition}[Eigenvalues and Eigenvectors]
	For $L$ a linear operator, a scalar $\lambda\in\mathbb C$ and nonzero ket $\ket{v}$ satisfying
	\[ L\ket{v} = \lambda\ket{v} 
	\]
	are called an \emph{eigenvalue} and its corresponding \emph{eigenvector}.
\end{definition}

Spectral theorems guarantee diagonalizability of Hermitian and normal operators.

\section{Tensor Products}
For composite quantum systems, state spaces combine via the tensor product.
\begin{definition}[Tensor Product of Spaces]
	Given Hilbert spaces $\mathcal H_A$ and $\mathcal H_B$, their tensor product $\mathcal H_A\otimes\mathcal H_B$ is the completion of the span of simple tensors $\ket{\psi}_A\otimes\ket{\phi}_B$ under the inner product
	\[ \braket{\psi_A\otimes\phi_B | \psi'_A\otimes\phi'_B} = \braket{\psi_A|\psi'_A}\,\braket{\phi_B|\phi'_B}. \]
\end{definition}

\begin{definition}[Tensor Product of Operators]
	For $A\in\mathcal L(\mathcal H_A)$ and $B\in\mathcal L(\mathcal H_B)$, define $A\otimes B\in\mathcal L(\mathcal H_A\otimes\mathcal H_B)$ by
	\[ (A\otimes B)(\ket{\psi}_A\otimes\ket{\phi}_B) = (A\ket{\psi}_A)\otimes(B\ket{\phi}_B). \]
\end{definition}

\section{Matrix Representations and Quantum Gates}
In an orthonormal basis $\{\ket{i}\}$, operators and kets admit matrix and column-vector representations. Common single-qubit gates include:
\[
X=\begin{pmatrix}0 & 1\\1 & 0\end{pmatrix},\quad
Y=\begin{pmatrix}0 & -i\\i & 0\end{pmatrix},\quad
Z=\begin{pmatrix}1 & 0\\0 & -1\end{pmatrix},\quad
H=\frac1{\sqrt2}\begin{pmatrix}1 & 1\\1 & -1\end{pmatrix}.
\]
Multi-qubit gates arise as tensor products of these.
\begin{definition}[Action of Single‐Qubit Gates on Canonical States]
	Let $\mathcal H=\operatorname*{span}\{\ket0,\ket1\}$ be the single‐qubit Hilbert space, and define
	\[
	\ket+\;=\;\tfrac1{\sqrt2}(\ket0+\ket1),
	\qquad
	\ket-\;=\;\tfrac1{\sqrt2}(\ket0-\ket1).
	\]
	The Pauli and Hadamard gates act on these four states as follows:
	\[
	\begin{aligned}
		X\ket0&=\ket1,&\quad X\ket1&=\ket0,\\
		X\ket+&=\ket+,&\quad X\ket-&=-\ket-,\\[6pt]
		Y\ket0&=i\ket1,&\quad Y\ket1&=-\,i\ket0,\\
		Y\ket+&=i\ket-,&\quad Y\ket-&=-\,i\ket+,\\[6pt]
		Z\ket0&=\ket0,&\quad Z\ket1&=-\ket1,\\
		Z\ket+&=\ket-,&\quad Z\ket-&=\ket+,\\[6pt]
		H\ket0&=\ket+,&\quad H\ket1&=\ket-,\\
		H\ket+&=\ket0,&\quad H\ket-&=\ket1.
	\end{aligned}
	\]
	In matrix form (in the $\{\ket0,\ket1\}$ basis),
	\[
	X=\begin{pmatrix}0&1\\1&0\end{pmatrix},\;
	Y=\begin{pmatrix}0&-i\\i&0\end{pmatrix},\;
	Z=\begin{pmatrix}1&0\\0&-1\end{pmatrix},\;
	H=\frac1{\sqrt2}\begin{pmatrix}1&1\\1&-1\end{pmatrix}.
	\]
\end{definition}

\begin{definition}[Hermitian Operator]
	An operator \(H\) on a Hilbert space \(\mathcal H\) is called \emph{Hermitian} (or self–adjoint) if
	\[
	H^\dagger = H.
	\]
	Equivalently, for all \(\ket{\psi},\ket{\phi}\in\mathcal H\),
	\[
	\bra{\psi}H\ket{\phi}
	= \overline{\bra{\phi}H\ket{\psi}}.
	\]
\end{definition}

\begin{definition}[Unitary Operator]
	An operator \(U\) on \(\mathcal H\) is called \emph{unitary} if
	\[
	U^\dagger U \;=\; U\,U^\dagger \;=\; I.
	\]
	Equivalently, \(U^{-1}=U^\dagger\), and \(U\) preserves inner products:
	\[
	\bra{U\psi}\,U\phi = \bra{\psi}\phi.
	\]
\end{definition}


\section{Conclusion}
This linear algebra toolkit underpins quantum algorithm design and analysis. Mastery of these concepts is essential for advanced study in quantum computation and information.


